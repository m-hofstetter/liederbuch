\documentclass[../main.tex]{subfiles}
\begin{document}
\begin{song}{Die freie Republik}{Traditionell}{}
[D]In dem Kerker saßen zu [A7]Frankfurt an dem [D]Main
[D]Schon seit vielen Jahren [A7]sechs Studenten [D]ein,
Die [G]für die Freiheit fochten und für das Bürger[D]glück
Und [D]für die Menschenrechte der [A7]freien Repu[D]blik.
Die [G]für die Freiheit fochten und für das Bürger[D]glück
Und [D]für die Menschenrechte der [A7]freien Repu[D]blik.

[D]Und der Kerkermeister [A7]sprach es täglich [D]aus:
\glqq{}[D]Sie, Herr Bürgermeister, es [A7]reißt mir keiner [D]aus.\grqq{}
Aber [G]doch sind sie verschwunden abends aus dem [D]Turm,
[D]Um die zwölfte Stunde [A7]bei dem großen [D]Sturm.
Aber [G]doch sind sie verschwunden abends aus dem [D]Turm,
[D]Um die zwölfte Stunde [A7]bei dem großen [D]Sturm.

[D]Und am ander'n Morgen [A7]hört man den Al[D]arm.
[D]Oh, es war entsetzlich, [A7]der Soldaten[D]schwarm.
Sie [G]suchten auf und nieder, sie suchten hin und [D]her.
Sie [D]suchten sechs Studenten und [A7]fanden sie nicht [D]mehr.
Sie [G]suchten auf und nieder, sie suchten hin und [D]her.
Sie [D]suchten sechs Studenten und [A7]fanden sie nicht [D]mehr.

[D]Doch sie kamen wieder mit [A7]Schwertern in der [D]Hand.
[D]Auf ihr deutschen Brüder, jetzt [A7]geht's für's Vater[D]land.
Jetzt [G]geht's für Menschenrechte und für das Bürger[D]glück.
Wir [D]sind doch keine Knechte der [A7]freien Repu[D]blik.
Jetzt [G]geht's für Menschenrechte und für das Bürger[D]glück.
Wir [D]sind doch keine Knechte der [A7]freien Repu[D]blik.

[D]Wenn euch die Leute fragen: \glqq{}[A7]Wo ist Absa[D]lom?\grqq{}
[D]So dürft ihr wohl sagen: \glqq{}[A7]Oh, der hänget [D]schon.
Er [G]hängt an keinem Baume, er hängt an keinem [D]Strick,
sond[D]ern an dem Glauben der [A7]freien Repu[D]blik.
Er [G]hängt an keinem Baume, er hängt an keinem [D]Strick,
Sond[D]ern an dem Traume der f[A7]reien Repub[D]lik.\grqq{}[A7]{\h}[D]{\h}

\end{song}
\end{document}