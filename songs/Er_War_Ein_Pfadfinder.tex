\documentclass[../main.tex]{subfiles}
\begin{document}
\begin{song}{Er war ein Pfadfinder}{Worte: Jörg Janke, Weise: aus \glqq{}Sixteen Tons\grqq{}, Urheber umstritten}{}
\gSec{Refrain}
Er war ein Pf[Em]adfinder, v[C]on kernigem Schl[B7]iff,
Er h[Em]ielt ihr die Treue, was k[C]einer begr[B7]iff.
So m[Em]anche Vereine, die l[Am]ockten ihn raus,
Doch die Pf[Em]adfinderkluft, ja die z[G7]og er nicht [B7]aus.
Doch die Pf[Em]adfinderkluft, ja die z[B7]og er nicht [Em]aus.

Mit zwölf J[Em]ahren fing er als J[C]ungpfadfinder [B7]an,
Er w[Em]ar zwar der kl[C]einste, aber ein M[B7]ann
Und [Em]alle Gesetze von B[Am]aden Powell,
Die k[Em]annte er schon damals v[B7]ery w[Em]ell.

\gSec{Refrain}
Er war ein Pf[Em]adfinder, v[C]on kernigem Schl[B7]iff,
Er h[Em]ielt ihr die Treue, was k[C]einer begr[B7]iff.
So m[Em]anche Vereine, die l[Am]ockten ihn raus,
Doch die Pf[Em]adfinderkluft, ja die z[G7]og er nicht [B7]aus.
Doch die Pf[Em]adfinderkluft, ja die z[B7]og er nicht [Em]aus.

Mit [Em]13 Jahren war er schon S[C]ippensuppenk[B7]och,
Vers[Em]alzte die S[C]uppe noch und n[B7]och,
Zwei Pf[Em]und Salz in der Suppe l[Am]ießen ihn kalt
Und [Em]er machte auch vor Regenw[B7]{ü}rmern nicht h[Em]alt.

\gSec{Refrain}
Er war ein Pf[Em]adfinder, v[C]on kernigem Schl[B7]iff,
Er h[Em]ielt ihr die Treue, was k[C]einer begr[B7]iff.
So m[Em]anche Vereine, die l[Am]ockten ihn raus,
Doch die Pf[Em]adfinderkluft, ja die z[G7]og er nicht [B7]aus.
Doch die Pf[Em]adfinderkluft, ja die z[B7]og er nicht [Em]aus.

Des N[Em]achts schlief er immer [C]unter dem B[B7]ett.
Die F[Em]olge davon w[C]ar, er wurde Korn[B7]ett.
Die S[Em]ippe kauft zum Lager Sch[Am]aumgummi ein,
Doch [Em]er schlief lieber auf Sch[B7]ottergest[Em]ein.

\gSec{Refrain}
Er war ein Pf[Em]adfinder, v[C]on kernigem Schl[B7]iff,
Er h[Em]ielt ihr die Treue, was k[C]einer begr[B7]iff.
So m[Em]anche Vereine, die l[Am]ockten ihn raus,
Doch die Pf[Em]adfinderkluft, ja die z[G7]og er nicht [B7]aus.
Doch die Pf[Em]adfinderkluft, ja die z[B7]og er nicht [Em]aus.\pagebreak

Mit [Em]17 trat er in die T[C]anzschule [B7]ein
Und tr[Em]at seiner P[C]artnerin öfter gegen das B[B7]ein.
Er w[Em]iegte die Mädchen im T[Am]angoschritt,
Doch d[Em]as Fahrtenmesser führte er im S[B7]ockenhalter m[Em]it.

\gSec{Refrain}
Er war ein Pf[Em]adfinder, v[C]on kernigem Schl[B7]iff,
Er h[Em]ielt ihr die Treue, was k[C]einer begr[B7]iff.
So m[Em]anche Vereine, die l[Am]ockten ihn raus,
Doch die Pf[Em]adfinderkluft, ja die z[G7]og er nicht [B7]aus.
Doch die Pf[Em]adfinderkluft, ja die z[B7]og er nicht [Em]aus.

Und als [Em]er endlich F[C]eldmeister w[B7]ar,
Da l[Em]iebte er ein M[C]{ä}dchen mit strohblondem H[B7]aar.
Er l[Em]iebte sie innig, doch sie w[Am]ar ihm nicht treu,
Da w[Em]idmete er sich wieder der Pf[B7]adfinder[Em]ei.

\gSec{Refrain}
Er war ein Pf[Em]adfinder, v[C]on kernigem Schl[B7]iff,
Er h[Em]ielt ihr die Treue, was k[C]einer begr[B7]iff.
So m[Em]anche Vereine, die l[Am]ockten ihn raus,
Doch die Pf[Em]adfinderkluft, ja die z[G7]og er nicht [B7]aus.
Doch die Pf[Em]adfinderkluft, ja die z[B7]og er nicht [Em]aus.

Am [Em]30. Mai kratzte [C]er sich am B[B7]ein.
Mit Bl[Em]utvergiftung g[C]ing er in die Jagdgründe [B7]ein.
Chief B[Em]aden Powell empfing ihn am H[Am]immelstor,
Zur B[Em]egrüßung sang ein [B7]englischer Ch[Em]or.

\gSec{Refrain}
Er war ein Pf[Em]adfinder, v[C]on kernigem Schl[B7]iff,
Er h[Em]ielt ihr die Treue, was k[C]einer begr[B7]iff.
So m[Em]anche Vereine, die l[Am]ockten ihn raus,
Doch die Pf[Em]adfinderkluft, ja die z[G7]og er nicht [B7]aus.
Doch die Pf[Em]adfinderkluft, ja die z[B7]og er nicht [Em]aus.

\end{song}
\end{document}