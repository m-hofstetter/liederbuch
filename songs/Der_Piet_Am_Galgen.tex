\documentclass[../main.tex]{subfiles}
\begin{document}
\begin{song}{Der Piet am Galgen}{Erik Martin}{}       
Was [Am]kann ich denn dafür? So [C]kurz vor meiner Tür,
da [G]fingen sie mich ein, und bald [Am]endet meine Pein.
Ich [Am]hatte niemals Glück. Mein [C]trostloses Geschick
nahm [G]keinen von euch ein, ja heut' [Am]soll gestorben sein.

\gSec{Refrain}
[Am]Wenn der Nebel auf das [Dm]Moor sich senkt,
der [F]Piet am [G]Galgen [Am]hängt.
[Am]Wenn der Nebel auf das [Dm]Moor sich senkt,
der [F]Piet am [G]Galgen [Am]hängt.

Sie [Am]nahmen mir die Schuh und [C]auch den Rock dazu,
sie [G]banden mir die Händ' und mein [Am]Haus, es hat gebrennt.
Ich [Am]sah den Galgen steh'n, sie [C]zwangen mich zu geh'n.
Sie [G]wollten meinen Tod, keiner [Am]half mir in der Not.

\gSec{Refrain}
[Am]Wenn der Nebel auf das [Dm]Moor sich senkt,
der [F]Piet am [G]Galgen [Am]hängt.
[Am]Wenn der Nebel auf das [Dm]Moor sich senkt,
der [F]Piet am [G]Galgen [Am]hängt.

Was [Am]kratzt da im Genick? Ich [C]spür' den rauen Strick.
Ein [G]Mönch, der betet dort und spricht [Am]für mich fromme Wort'.
Die [Am]Wort', die ich nicht kenn', wer [C]lehrte sie mich denn?
Fünf [G]Raben fliegen her, doch ich [Am]sehe sie nicht mehr.

\gSec{Refrain}
[Am]Wenn der Nebel auf das [Dm]Moor sich senkt,
der [F]Piet am [G]Galgen [Am]hängt.
[Am]Wenn der Nebel auf das [Dm]Moor sich senkt,
der [F]Piet am [G]Galgen [Am]hängt.
\end{song}
\end{document}