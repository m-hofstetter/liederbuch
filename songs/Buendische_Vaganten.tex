\documentclass[../main.tex]{subfiles}
\begin{document}
\begin{song}{Bündische Vaganten}{Trenk (Alo Hamm)}{\glqq Ayen\grqq{} ist der traditionelle Zugvogel-Gruß, wie das \glqq Gut Pfad\grqq{} der Pfadfinder.}
[Em]Hej, wie vorn der [B7]Fetzen fliegt,
Hej, [Em]wie er sich im [B7]Winde wiegt
[Am]Ohne [Em]Rast und [B7]ohne [Em]Ruh.
So [Em]wiegen wir mit [B7]freiem Sinn
Uns [Em]{ü}ber tausend [B7]Strassen hin,
[Am]ohne [Em]Ende, [B7]immer[Em]zu.

\gSec{Refrain}
[C]Bündische Va[G]ganten
[D7]Tippeln in die Welt, [G]tippeln in die Welt.
[C]Bündische Va[G]ganten
[D7]Tippeln in die Welt. Hey o A[G]yen!

[Em]Treiben wir dem [B7]Süden zu,
Lässt [Em]uns der Norden [B7]keine Ruh,
[Am]{Ü}ber[Em]all zu [B7]Haus sind [Em]wir.
Mal [Em]rüber nach [B7]Amerika,
Mal [Em]runter bis nach [B7]Afrika
[Am]Hoja, [Em]hoja, [B7]das sind [Em]wir.

\gSec{Refrain}
[C]Bündische Va[G]ganten
[D7]Tippeln in die Welt, [G]tippeln in die Welt.
[C]Bündische Va[G]ganten
[D7]Tippeln in die Welt. Hey o A[G]yen!

[Em]Hast du noch ein [B7]jung' Gesicht,
So [Em]zage nicht und [B7]fack'le nicht,
[Am]Frage [Em]niemals [B7]nach dem [Em]wie!
Wer [Em]nur am Rand der [B7]Strasse klebt,
Für [Em]seinen dummen [B7]Bauche lebt
[Am]Misst der [Em]Ferne [B7]Zauber [Em]nie!

\gSec{Refrain}
[C]Bündische Va[G]ganten
[D7]Tippeln in die Welt, [G]tippeln in die Welt.
[C]Bündische Va[G]ganten
[D7]Tippeln in die Welt. Hey o A[G]yen!

\end{song}
\end{document}