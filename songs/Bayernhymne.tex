\documentclass[../main.tex]{subfiles}
\begin{document}
\begin{song}[2]{Bayernhymne}{Worte:  Michael Öchsner, Weise: Konrad Max Kunz}{Die dritte Strophe mit Bezug zu Europa und Gleichberechtigung der Geschlechter von Muhammed Agca, Tatjana Sommerfeld und Benedikt Kreisl ist nicht Teil der offiziellen Hymne. Die Erweiterung des Liedes wurde 2012 durch den Landtag abgelehnt.}
[G]Gott mit [D7]dir, du [C]Land der [D7]Bayern, [G]deutsche [D7]Erde, [C]Va[D7]terl[G]and!
Über [G]deinen [Am]weiten [G]Gauen [Am]ruhe [D7]seine Se[C]gensh[G]and!

[D7]Er be[Am]hüte deine Fl[D7]uren, schirme [Am]deiner Städte [D]Bau
[D7]Und er[G]halte dir die [C]Farben seines [D7]Himmels, weiß und bl[G]au!
[D7]Er be[Am]hüte deine Fl[D7]uren, schirme [Am]deiner Städte [D]Bau
[D7]Und er[G]halte dir die [C]Farben seines [D7]Himmels, weiß und bl[G]au!

[G]Gott mit [D7]dir, dem [C]Bayern[D7]volke, [G]dass wir, [D7]uns'rer [C]Vä[D7]ter [G]wert,
fest in [G]Eintracht [Am]und in [G]Frieden [Am]bauen [D7]uns'res [C]Glückes [G]Herd!

[D7]Dass mit [Am]Deutschlands Bruders[D7]tämmen einig [Am]uns ein jeder [D]schau
[D7]und den [G]alten Ruhm bew[C]{ä}hre; unser [D7]Banner, weiß und [G]blau!
[D7]Dass mit [Am]Deutschlands Bruders[D7]tämmen einig [Am]uns ein jeder [D]schau
[D7]und den [G]alten Ruhm bew[C]{ä}hre; unser [D7]Banner, weiß und [G]blau!

\gSec{Inoffizielle Strophe}
[G]Gott mit [D7]uns und [C]allen [D7]Völkern, [G]ganz in [D7]Einheit [C]tun [D7]wir [G]kund:
In der [G]Vielfalt [Am]liegt die [G]Zukunft, [Am]in Eur[D7]opas Staa[C]tenb[G]und.

[D7]Freie [Am]Menschen, freies [D7]Leben, gleiches [Am]Recht für Mann und [D]Frau!
[D7]Goldne [G]Sterne, blaue [C]Fahne und der [D7]Himmel, weiß und [G]blau!
[D7]Freie [Am]Menschen, freies [D7]Leben, gleiches [Am]Recht für Mann und [D]Frau!
[D7]Goldne [G]Sterne, blaue [C]Fahne und der [D7]Himmel, weiß und [G]blau!
\end{song}
\end{document}