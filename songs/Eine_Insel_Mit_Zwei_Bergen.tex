\documentclass[../main.tex]{subfiles}
\begin{document}
\begin{song}{Eine Insel mit zwei Bergen}{Worte: Manfred Jenning, Weise: Hermann Amann}{}\zusatzeintrag{Lummerlandlied}
[G]Ei[D]ne [G]Insel mit zwei Bergen und im [D]tiefen, weiten Meer
Mit viel [D7]Tunnels und Geleisen und dem [G]Eisenbahnverkehr
[G]Nun, wie mag die Insel heißen, ringshe[D]rum ist schöner Strand
Jeder [D7]sollte einmal reisen, in das sc[G]höne Lummerland!

[G]Ei[D]ne [G]Insel mit zwei Bergen und dem [D]Fotoatelier
In dem [D7]letzten macht man Bilder, auf den [G]ersten \glqq{}Dulijö\grqq{}
[G]Diese Breiten, diese Tiefen, diese [D]Höhen sind bekannt
Und man [D7]spricht von den Motiven auf dem [G]schönen Lummerland!

[G]Ein[D]e I[G]nsel mit zwei Bergen und dem [D]Fernsprechtelefon
wählt man [D7]nur die richtige Nummer, klappt auch die V[G]erbindung schon.
[G]\glqq{}Hallo, hier ist falsch verbunden!\grqq{} -- \glqq{}Wollen [D]sie sich jetzt beschwer'n?\grqq{}
\glqq{}Nein, [D7]warum? Das kann passier'n!\grqq{} -- \glqq{}Also d[G]ann auf Wiederhör'n!\grqq{}

[G]Ein[D]e I[G]nsel mit zwei Bergen und dem [D]Laden von Frau Waas:
Husten[D7]bonbons, Alleskleber, Regen[G]schirme, Leberkas,
[G]Körbe, Hüte, Lampen, Würste, Blumen[D]kohl und Fensterglas,
Leder[D7]hosen, Kukucksuhren und noch [G]dies und dann noch das!
\end{song}
\end{document}