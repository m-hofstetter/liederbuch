\documentclass[../main.tex]{subfiles}
\begin{document}
\begin{song}{Die alten Rittersleut'}{Karl Valentin}{}\zusatzeintrag{Ja so warn's}
Zu [C]Grünewald im Isartal, [Am]glaubt es mir, es war einmal,
d[Dm]a ham edle Ritter g'haust, [G]ja die h[G7]at's vor g[C]ar nix graust.

Ja, so wa[C]rn's, ja, so warn's, ja, so warn's die a[F]lten R[G]ittersleut, ja, so w[C]arn's, ja, so warn's, die a[G]lten R[G7]ittersl[C]eut.

S[C]o ein alter Rittersmann, h[Am]atte sehr viel Eisen an,
die m[Dm]eisten Ritter, i muß sagn, h[G]at desh[G7]alb der Bl[C]itz erschlagn.

Ja, so wa[C]rn's, ja, so warn's, ja, so warn's die a[F]lten R[G]ittersleut, ja, so w[C]arn's, ja, so warn's, die a[G]lten R[G7]ittersl[C]eut.

G's[C]uffa hams and des net wia, a[Am]us die Eimer Wein und Bier,
h[Dm]ams dann alles suffa ghabt, sans u[G]nterm T[G7]isch dr[C]unt g'flaggt.

Ja, so wa[C]rn's, ja, so warn's, ja, so warn's die a[F]lten R[G]ittersleut, ja, so w[C]arn's, ja, so warn's, die a[G]lten R[G7]ittersl[C]eut.

H[C]att der Ritter ein Katarrh, d[Am]amals warn die Mittel rar,
e[Dm]r hat der Erkältung trotzt, s[G]ich ger[G7]{ä}uspert u[C]nd geschneuzt.

Ja, so wa[C]rn's, ja, so warn's, ja, so warn's die a[F]lten R[G]ittersleut, ja, so w[C]arn's, ja, so warn's, die a[G]lten R[G7]ittersl[C]eut.

R[C]itt ein Ritter auf seim Roß, w[Am]ar das Risiko sehr groß,
h[Dm]at ein Roß sein Hupfer do, l[G]ag im Dr[G7]eck der g[C]ute Mo.

Ja, so wa[C]rn's, ja, so warn's, ja, so warn's die a[F]lten R[G]ittersleut, ja, so w[C]arn's, ja, so warn's, die a[G]lten R[G7]ittersl[C]eut.

U[C]nd der Ritter Kunibart, s[Am]etzte sich verkehrt aufs Pferd,
w[Dm]ollte er nach hinten sehn, br[G]aucht er s[G7]ich nicht u[C]mzudrehn.

Ja, so wa[C]rn's, ja, so warn's, ja, so warn's die a[F]lten R[G]ittersleut, ja, so w[C]arn's, ja, so warn's, die a[G]lten R[G7]ittersl[C]eut.

U[C]nd das Ritterfräulein Stasi, w[Am]ar so bleich und schrecklich kasig,
w[Dm]ar ihr mal ein Knecht zuwider, s[G]enkt sie b[G7]arsch die A[C]ugenlieder.

Ja, so wa[C]rn's, ja, so warn's, ja, so warn's die a[F]lten R[G]ittersleut, ja, so w[C]arn's, ja, so warn's, die a[G]lten R[G7]ittersl[C]eut.

S[C]o ein früheres Ritterweib w[Am]ar dem Ritter niemals treu,
dem R[Dm]itter war es einerlei, er w[G]ar ja a[G7]uch nur h[C]albe treu.

Ja, so wa[C]rn's, ja, so warn's, ja, so warn's die a[F]lten R[G]ittersleut, ja, so w[C]arn's, ja, so warn's, die a[G]lten R[G7]ittersl[C]eut.

U[C]nd der Ritter Alexander r[Am]utsche einst auf dem Gelander,
u[Dm]nten stand ein Nagel vor, h[G]eut singt [G7]er im Kn[C]abenchor.

Ja, so wa[C]rn's, ja, so warn's, ja, so warn's die a[F]lten R[G]ittersleut, ja, so w[C]arn's, ja, so warn's, die a[G]lten R[G7]ittersl[C]eut.

Zu Gr[C]{ü}newalt die Rittersleut le[Am]b'n nicht mehr seit langer Zeit,
n[Dm]ur die Geister von densölben sp[G]uken n[G7]achts in d[C]en Gewölben.

Ja, so wa[C]rn's, ja, so warn's, ja, so warn's die a[F]lten R[G]ittersleut, ja, so w[C]arn's, ja, so warn's, die a[G]lten R[G7]ittersl[C]eut.

\end{song}
\end{document}