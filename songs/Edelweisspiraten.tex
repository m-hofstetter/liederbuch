\documentclass[../main.tex]{subfiles}
\begin{document}
\begin{song}{Edelweißpiraten}{Herwig Steymans}{Als \glqq Edelweißpiraten\grqq{} werden verschiedene Jugendgruppen bezeichnet, die zur Zeit des Nationalsozialismus aus der verbotenen Bündischen Jugend hervorgegangen waren. Die Edelweißpiraten standen in Feindschaft mit der Hitler-Jugend und wurden gewaltsam durch die Gestapo verfolgt. Einige dieser Gruppen engagierten sich unter dem Einsatz ihres Lebens im Kampf gegen die Nazis und versteckten zum Beispiel Juden und Deserteure. Die Edelweißpiraten wurden auch nach dem Ende des NS-Regimes noch von den deutschen Behörden kriminalisiert, da diese oft immer noch mit ehemaligen Gestapo-Beamten besetzt waren.}

[G]Sie saßen [D]oft beim Märche[C]nsee am Lage[G]rfeuer,
[Em]sie wollten [Am]leben, wie es [D]ihnen ge[G]fiel.
[G]Der neue [D]Kurs im deutschen [C]Reich war nicht [G]geheuer.
[Em]sie wollten [Am]frei sein mit Ge[D]sang, Gitarren[G]spiel.
[D]Mit ihrer [C]Kleidung nahmen [G]sie's nicht so ge[D]nau,
ganz offen [C]trugen sie das [G]Edelweiß zur [D]Schau
sie hatten M[C]ut und das war g[G]ut.

\gSec{Refrain}
[G]Vielleicht wird [D]morgen schon[C] eine neue [G]Zeit anfangen,
vielleicht ist [D]morgen schon de[C]r Spuk vor[G]bei.
[G]Vielleicht wird [D]morgen schon[C] eine neue [G]Zeit anfangen,
vielleicht ist [D]morgen schon de[C]r Spuk vor[G]bei.

[G]Sie hatten [D]nichts im Sinn von [C]braunen Nazi[G]horden,
[Em]sie hielten [Am]nichts von dem Ge[D]schrei nach Heil und [G]Sieg.
[G]Was war denn [D]nur aus ihrem [C]Vaterland ge[G]worden?
[Em]Man schürte [Am]offen den ver[D]brecherischen [G]Krieg.
[D]Da gab's nur [C]eins zu tun: Be[G]frein' wir dieses [D]Land,
da durfte [C]keiner ruhn': Wir [G]leisten Wider[D]stand!
Sie hatten [C]Mut und das war [G]gut.

\gSec{Refrain}
[G]Vielleicht wird [D]morgen schon[C] eine neue [G]Zeit anfangen,
vielleicht ist [D]morgen schon de[C]r Spuk vor[G]bei.
[G]Vielleicht wird [D]morgen schon[C] eine neue [G]Zeit anfangen,
vielleicht ist [D]morgen schon de[C]r Spuk vor[G]bei.

[G]Da gab's 'nen [D]Güterzug mit [C]Kriegsgerät und [G]Waffen
[Em]und was man [Am]sonst noch braucht für [D]einen Völker[G]mord.
[G]Da machten [D]sie sich an den [C]Gleisen kurz zu [G]schaffen,
[Em]der Zug er[Am]reichte niemals [D]den Bestimmungs[G]ort.
[D]Und Essens[C]marken vom Par[G]teibüro der [D]Stadt,
waren plötzlich [C]weg und Zwangsar[G]beiter wurden [D]satt.
Sie hatten [C]Mut und das war [G]gut.\pagebreak

\gSec{Refrain}
[G]Vielleicht wird [D]morgen schon[C] eine neue [G]Zeit anfangen,
vielleicht ist [D]morgen schon de[C]r Spuk vor[G]bei.
[G]Vielleicht wird [D]morgen schon[C] eine neue [G]Zeit anfangen,
vielleicht ist [D]morgen schon de[C]r Spuk vor[G]bei.

[G]Sie glaubten [D]fest daran, dass [C]sie den Sieg er[G]ringen,
[Em]sie glaubten [Am]fest daran: aus [D]Schaden wird man [G]klug.
[G]Sie glaubten [D]fest dran als [C]sie zum Galgen [G]gingen.
[Em]Sie glaubten [Am]fest daran als [D]man sie vorher [G]schlug.
[D]Und diese [C]Angst, die hinter [G]jeder Folter [D]steckt,
die ist so [C]groß, daß, man den [G]besten Freund ver[D]rät.
Versteht man [C]gut, versteht man [G]gut.

\gSec{Refrain}
[G]Vielleicht wird [D]morgen schon[C] eine neue [G]Zeit anfangen,
vielleicht ist [D]morgen schon de[C]r Spuk vor[G]bei.
[G]Vielleicht wird [D]morgen schon[C] eine neue [G]Zeit anfangen,
vielleicht ist [D]morgen schon de[C]r Spuk vor[G]bei.

[G]Sie stehen [D]heute noch auf [C]vielen schwarzen [G]Listen.
[Em]Ich möchte [Am]sagen es ist [D]wieder mal so[G]weit.
[G]In Amt und [D]Würden sitzen [C]wieder mal Fa[G]schisten.
[Em]Und zum [Am]totalen Krieg ist [D]mancher schnell [G]bereit.
[D]Doch gibt es [C]einige - und [G]das beruhigt mich [D]sehr -
die zeigen [C]offen das zer[G]brochene Ge[D]wehr
Und das macht M[C]ut, und das macht Mu[G]t.

\gSec{Refrain}
[G]Vielleicht wird [D]morgen schon[C] eine neue [G]Zeit anfangen,
vielleicht ist [D]morgen schon de[C]r Spuk vor[G]bei.
[G]Vielleicht wird [D]morgen schon[C] eine neue [G]Zeit anfangen,
vielleicht ist [D]morgen schon de[C]r Spuk vor[G]bei.

\end{song}
\end{document}