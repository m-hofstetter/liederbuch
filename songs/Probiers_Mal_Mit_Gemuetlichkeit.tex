\documentclass[../main.tex]{subfiles}
\begin{document}
\begin{song}{Probier's mal mit Gemütlichkeit}{dt. Worte: Heinrich Riethmüller, Weise: Terry Gilkyson}{}
\gSec{Refrain}
Probier’s mal[D] mit [D7]Gemütlichkeit,
mit [G]Ruhe und Ge[G7]mütlichkeit wirfst [D]du die dummen [B7]Sorgen über [E7]Bord.[A]{\hh}
Und wenn du [D]stets ge[D7]mütlich bist und [G]etwas appe[G7]titlich ist,
greif [D]zu, denn [B7]später [E7]ist es vie[A7]lleicht [D]fort.

Was soll ich [A7]woanders, wo's mir nicht ge[D]fällt?
Ich gehe nicht [A7]fort hier, auch nicht für [D]Geld[D7].
Die Bienen [G]summen in der [Gm]Luft, erfüllen [D]sie mit Honig[E7]duft.
Und [Bm7]schaust du unter’nen [B7]Stein, entdeckst du [Em]Ameisen,
[B7]die hier gut [Em]gedeih’n. [A]Nimm [D]da von zwei, drei [B7]vier.
Denn mit Ge[Em]mütlichkeit kommt [A7]auch das Glück zu [D]dir! [Em]Es [A7]kommt zu [D]dir!

\gSec{Refrain}
Probier’s mal[D] mit [D7]Gemütlichkeit,
mit [G]Ruhe und Ge[G7]mütlichkeit wirfst [D]du die dummen [B7]Sorgen über [E7]Bord.[A]{\hh}
Und wenn du [D]stets ge[D7]mütlich bist und [G]etwas appe[G7]titlich ist,
greif [D]zu, denn [B7]später [E7]ist es vie[A7]lleicht [D]fort.

Na, und pflückst du gern [A7]Beeren und du piekst dich [D]dabei,
dann lass dich be[A7]lehren; Schmerz geht bald [D]vorbei![D7]{\hh}
Du musst bes[G]cheiden und nicht [Gm]gierig [D]im Leben [E7]sein,
sonst [Bm7]tust du dir weh, du [B7]bist verletzt und [Em]zahlst nur drauf,
[B7]drum pflücke gleich mit dem [Em]richt’gen Dreh!
[A]Hast du das [D]jetzt kap[B7]iert?
Denn mit Ge[Em]mütlichkeit kommt [A7]auch das Glück zu [D]dir! [Em]Es [A7]kommt zu [D]dir!

\gSec{Refrain}
Probier’s mal[D] mit [D7]Gemütlichkeit,
mit [G]Ruhe und Ge[G7]mütlichkeit wirfst [D]du die dummen [B7]Sorgen über [E7]Bord.[A]{\hh}
Und wenn du [D]stets ge[D7]mütlich bist und [G]etwas appe[G7]titlich ist,
greif [D]zu, denn [B7]später [E7]ist es vie[A7]lleicht [D]fort.
\end{song}
\end{document}