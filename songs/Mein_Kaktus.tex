\documentclass[../main.tex]{subfiles}
\begin{document}
\begin{song}{Mein kleiner grüner Kaktus}{Comedian Harmonists}{}
[A]Blumen im [E7]Garten, [A]so zwanzig [E7]Arten
[A]von Rosen, Tulpen und Narzi[F#]ssen,
[Bm]leisten sich [F#]heute [Bm]die feinen [F#]Leute.
[B7]Das will ich alles gar nicht [E7]wissen.

\gSec{Refrain}
Mein [A]kleiner grüner Kaktus steht draußen am Bal[E]kon, holari, holari, hola[A]ro!
Was brauch’ ich rote Rosen, was brauch’ ich roten [E]Mohn, holari, holari, hola[A]ro!
Und [D]wenn ein Bösewicht was [A]Ungezog’nes spricht,
dann [B7]hol’ ich meinen Kaktus und der [E7]sticht, sticht, sticht.
Mein [A]kleiner grüner Kaktus steht draußen am Bal[E]kon, holari, holari, holar[A]o!

[A]Man find’t [E7]gewöhnlich [A]die Frauen [E7]{ä}hnlich
[A]den Blumen, die sie gerne [F#]haben.
[Bm]Doch ich sag’ [F#]täglich: [Bm]Das ist [F#]unmöglich!
[B7]Was soll’n die Leut’ sonst vo[E7]n mir sagen?!

\gSec{Refrain}
Mein [A]kleiner grüner Kaktus steht draußen am Bal[E]kon, holari, holari, hola[A]ro!
Was brauch’ ich rote Rosen, was brauch’ ich roten [E]Mohn, holari, holari, hola[A]ro!
Und [D]wenn ein Bösewicht was [A]Ungezog’nes spricht,
dann [B7]hol’ ich meinen Kaktus und der [E7]sticht, sticht, sticht.
Mein [A]kleiner grüner Kaktus steht draußen am Bal[E]kon, holari, holari, holar[A]o!

[A]Heute um [E7]viere klopft’s [A]an die T[E7]{ü}re,
[A]nanu, Besuch so früh am [F#]Tage?
[Bm]Es war Herr [F#]Krause [Bm]vom Nach[F#]barhause,
[B7]er sagt: Verzeih’n Sie w[E7]enn ich frage.

\gSec{Refrain}
Sie [A]hab’n da doch einen Kaktus da draußen am Bal[E]kon, holari, holari, hola[A]ro!
Der fiel soeben runter, was halten Sie da[E]von? Holari, holari, hola[A]ro!
Er [D]fiel mir auf’s Gesicht, ob Sies’ [A]glauben oder nicht,
jetzt [B7]weiß ich, dass Ihr kleiner [E7]grüner Kaktus sticht!
Bewah[A]r’n Sie ihren Kaktus gefälligst and[E]erswo, holari, holari, hola[A]ro!
\end{song}
\end{document}