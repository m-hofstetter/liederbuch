\usepackage[margin=20mm]{geometry}
\usepackage[ngerman]{babel}
\usepackage[T1]{fontenc}
\usepackage[utf8]{inputenc}

\usepackage{guitar}

\usepackage{hyperref}
\usepackage{xcolor}
\usepackage[nonewpage, xindy]{imakeidx}

\usepackage{fancyhdr}
\usepackage{titlesec}
\usepackage{graphicx}
\usepackage{ifthen}
\usepackage{tocloft}
\usepackage{geometry}
\usepackage{xifthen}
\usepackage{tikz}
\usepackage{sty/simplegtab}
\usepackage{sty/simplesong}
\usepackage{multicol}
\usepackage{subfiles}


% Keine Seitenzahlen im TOC
\let\Contentsline\contentsline
\renewcommand\contentsline[3]{\Contentsline{#1}{#2}{#3}}

\makeindex[program=texindy, columns=2, title=Alphabetisches Inhaltsverzeichnis, options= -L german-din -M sty/index_style.xdy, columnsep=1.2cm]


\titleformat{\section}[hang]{\normalfont\Large\bfseries}{}{0pt}{} % Keine Nummerierung fuer section
\raggedbottom % text wird nicht vertikal ausgedehnt
\def\guitarPreAccord{\footnotesize\strut\sffamily\color{gray}\bfseries} % Formatierung Akkorde
\newcommand{\h}{\hspace{12mm}} % Abstand Soloakkorde
\newcommand{\hh}{\hspace{6mm}} % halber Abstand Soloakkorde 
\newcommand{\gSec}[1]{{\bfseries #1}\nopagebreak} % Formatierung Chorus, Intro etc.
\newcommand{\sprache}[1]{\textit{#1}} % Formatierung Sprache
\newcommand{\zusatzeintrag}[1]{\index{#1@\textit{#1}}}
\newcommand{\abschnitt}[1]{\addcontentsline{toc}{part}{#1}\fancyhead[C]{#1}}
\cftpagenumbersoff{part}

% Kopf- und Fusszeile
\pagestyle{fancy}
\fancyhead{}
\renewcommand{\headrulewidth}{0.4pt}
\renewcommand{\footrulewidth}{0.4pt}

% TOC
\renewcommand{\cftsecfont}{\normalfont}
\setlength{\cftbeforesecskip}{0mm}
\cftsetindents{section}{0mm}{5mm}
\renewcommand{\cftsecnumwidth}{2mm}
\renewcommand{\cftsecpagefont}{\normalfont}
\renewcommand{\cftsecaftersnum}{. \hfill}
\setlength{\columnsep}{1.2cm}
\renewcommand{\cftpnumalign}{r}
\addto\captionsngerman{
  \renewcommand{\contentsname}
    {}
}
\renewcommand{\numberline}[1]{}

\fancypagestyle{plain}{%
  \fancyhf{}
  \pagestyle{fancy}\fancyfoot{}\fancyfoot[C]{Inhaltsverzeichnis}
}